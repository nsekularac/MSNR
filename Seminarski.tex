% !TEX encoding = UTF-8 Unicode

\documentclass[a4paper]{article}

\usepackage{color}
\usepackage{url}
\usepackage[T2A]{fontenc} % enable Cyrillic fonts
\usepackage[utf8]{inputenc} % make weird characters work
\usepackage{graphicx}
\usepackage{physics}

\usepackage[english,serbian]{babel}
%\usepackage[english,serbianc]{babel} %ukljuciti babel sa ovim opcijama, umesto gornjim, ukoliko se koristi cirilica

\usepackage[unicode]{hyperref}
\hypersetup{colorlinks,citecolor=green,filecolor=green,linkcolor=blue,urlcolor=blue}

%\newtheorem{primer}{Пример}[section] %ćirilični primer
\newtheorem{primer}{Primer}[section]

\begin{document}

\title{Kvantno računarstvo 2 (funkcionalni programski jezici za kvantno programiranje)\\ \small{Seminarski rad u okviru kursa\\Metodologija stručnog i naučnog rada\\ Matematički fakultet}}

\author{Aleksandar Ćurković, Nemanja Šekularac\\ curkovical@gmail.com, nemanja.sekularac@gmail.com}
\date{14.~april 2015.}
\maketitle

\abstract{
U ovom radu prikazujemo funkcionalne programske jezike za kvantno programiranje, uz kratak osvrt na osnove kvantnog računarstva i funkcionalne programske jezike uopšte.
Propratni primeri realizovani su u programskom jeziku Quipper.

\tableofcontents

\newpage

\section{Uvod}
\label{sec:uvod}

\subsection{Kvantno računarstvo}
\label{sec:kvantnoracunarstvo}

Kvantno računarstvo je oblast istraživanja koja se bavi računarskim sistemima koji u svom radu direktno koriste efekte kvantne mehanike.
Kvantni efekti, kao što su superpozicija (eng.~\em{superposition}) i kvantna zapletenost (eng. ~\em{quantum entanglement}) se u ovim sistemima, koje nazivamo kvantnim računarima tretiraju kao resurs i koriste za izvršavanje operacija nad podacima.

\subsection{Osnovni pojmovi kvantnog računarstva}

\subsubsection {Kjubit}

Kjubit (eng.~\emph{qubit, quantum bit}) je najmanja jedinica izračunavanja u kvantnom računarstvu.

Kjubit je pandan jednom bitu informacija u klasičnom računarstvu, ali dok se jedan bit može naći u 2 diskretna stanja (0, 1),
kjubit, na osnovu kvantnog principa superpozicije može istovremeno biti u oba.
Tačnije, stanje kjubita, u odsustvu merenja, može biti bilo koja linearna kombinacija osnovnih stanja.
Za osnovna stanja (eng. ~\em base states). kjubita se uzimaju ona koja pri merenju prelaze u stanje 0 ili 1 sa verovatnoćom 1 i obeležavaju se sa $\ket{1}$ i $\ket{0}$

Formalno, stanje kjubita se zapisuje kao $\alpha \ket{0} + \beta \ket{1}$ gde su $\alpha i \beta$ \em{kompleksni} parametri koji označavaju verovatnoću da se nakon merenja (očitavanja stanja) kjubit nađe u jednom od osnovnih stanja sa verovatnoćama, redom, $\alpha^2$ i $\beta^2$ i važi $\alpha^2+\beta^2=1$

Ovo svojstvo ima za posledicu da se 2 kubita mogu naći istovremeno u 4 različita stanja i generalno, n kubita može odjednom biti u $2^n$ različitih stanja.

Skup kjubita uzetih zajedno naziva se kvantni registar. (eng.~\emph{quantum register}).

\subsubsection{Zapletenost, kombinovanje kjubita}
\label{entanglement}

Efekat kvantne zapletenosti se manifestuje visoko koreliranim stanjima zapletenih čestica nakon merenja.
To nam omogućava da kombinujemo kjubite na način na koji to nije moguće sa bitovima. Stanje sistema od 2 zapletena kjubita predstavlja se kao linearna kombinacija svih kombinacija osnovnih stanja oba elementa.\\

$\alpha\ket{00} + \beta\ket{01} + \gamma\ket{10} + \delta\ket{11}$\\

pri čemu su $\alpha^2, \beta^2, \gamma^2, \delta,$ redom verovatnoće da će se sistem naći u odgovarajućoj kombinaciji osnovnih stanja gde i dalje važi\\

$\alpha^2+\beta^2+\gamma^2+\delta^2=1$

\subsubsection {Kvantna kapija, operacije nad kjubitima}

Kvantna logička kapija (eng. ~\emph{quantum logic gate}) je kvantni pandan logičkoj kapiji klasičnog računarstva.
Pokazano je da zbog očuvanja koherencije sistema, sve operacije nad kjubitima i sistemima kjubita moraju biti reverzibilne (eng.~\emph{reversible}) \cite{basics}

Očitavanje stanja predstavlja merenje koje u kvantnoj mehanici dovodi do dekoherencije (eng.~\emph{decoherence}) tj rušenja sistema iz superpozicije u kombinaciju osnovnih stanja i gubitka informacija.\\

Ovo nas dovodi do toga da osnovni oblik kvantnog algoritma uzima formu:\\

1. inicijalizacija stanja --> 2. unitarna operacija --> 3. merenje --> 4. provera rešenja --> 5. ponavljanje\\

Pri tom je operacija 2 proizvod unitarnih operacija od kojih se sastoji naš algoritam.

Treba imati u vidu da kvantni računari dele osobine sa probabilističkim i nedeterminističkim računarima i da tačno rešenje daju samo sa određenom tačnošću.

\subsubsection{Modeli kvantnog računara, Kvantno kolo}

Praktična konstrukcija kvantnog računara je još uvek u domenu laboratorijskih eksperimenata uz tek poneki komercijalni pokušaj.
Za potrebe kvantnog računarstva, sisteme predstavljamo različitim idealizovanim modelima kao što su npr kvantna Turingova mašina (eng.~\emph{quantum Turing machine, QTM}) ili kvantno kolo (eng. ~\emph{quantum circuit}). 

Kvantno kolo je sekvenca kvantnih kapija kao osnovnih operacija koje vrše transformacije nad jednim ili više kjubita.

\subsection{Funkcionalni programski jezici}
\label{sec:funkcionalniprgjezici}

\section{Funkcionalni programski jezici za kvantno programiranje}
\label{sec:funcprl_qp}


\subsection{Quantum Lambda Calculus}
\label{sec:lambdacalc}


\subsection{QFC and QPL}
\label{sec:qfc_qml}

QFC, QPL

\subsection{QML}
\label{sec:qml}

QML

\subsection{Quipper}
\label{sec:quipper}

Quipper je funkcionalni programski jezik predstavljen 2013. godine. 

\subsubsection{QRAM Model}
\subsubsection{Sintaksa}
\subsubsection{Izvršavanje na klasičnim računarima}
\label{sec:quipperizvrsavanje}

Quipper je realizovan kao ~\em embedded ~\em language u okviru funkcionalnog jezika Haskell sa dodatnim bibliotekama. Aktuelna distribucija sadrži i simulator kvantnog računara i kod pisan u quipperu se može, uz izvesna ograničenja izvršiti i na klasičnim računarima.


\subsection{Primeri}

\section{Zaključak}
\label{sec:zakljucak}




\addcontentsline{toc}{section}{Literatura}
\appendix
\bibliography{seminarski} 
\bibliographystyle{plain}


\end{document}
