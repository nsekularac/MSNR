% !TEX encoding = UTF-8 Unicode

\documentclass[a4paper]{report}

\usepackage[T2A]{fontenc} % enable Cyrillic fonts
\usepackage[utf8x,utf8]{inputenc} % make weird characters work
\usepackage[serbian]{babel}
%\usepackage[english,serbianc]{babel}
\usepackage{amssymb}

\usepackage{color}
\usepackage{url}
\usepackage[unicode]{hyperref}
\usepackage{dirtytalk}
\hypersetup{colorlinks,citecolor=green,filecolor=green,linkcolor=blue,urlcolor=blue}

\newcommand{\odgovor}[1]{\textcolor{blue}{#1}}

\begin{document}

\title{Kvantno računarstvo 2 (funkcionalni programski
jezici za kvantno programiranje)\\ \small{Aleksandar Ćurković, Nemanja Šekularac}}

\maketitle
\tableofcontents

\chapter{Prvi recenzent \odgovor{--- ocena: 4} }

\section{O čemu rad govori?}
Rad govori o osnovnim konceptima kvantnog računarstva. Objašnjava pojmove kao što su kjubit koji počiva na principu superpozicije, kvantni paralelizam, kvantno kolo i model kvantnog računara. Opisane su glavne karakteristike funkcionalnih jezika u kvantnom računarstvu na primeru jezika Quipper koji podržava simulaciju na klasičnom računaru.

\section{Krupne primedbe i sugestije}
Sažetak bi trebalo da se proširi. Trebalo bi reći neke osnovne karakteristike ili prednosti kvantnog računarstva.

Radu nedostaje sam uvod u rad, kreće se sa uvodom u kvantno računarstvo.

Zaključak bi trebalo da se preradi jer govori o tome šta je urađeno u radu a ne generalno u kvantnom računarstvu i postojećim funkcionalnim programskim jezicima.\\
\odgovor {Slažemo se s ovom ocenom, tako da smo zaključak dopunili: 
\say{S obzirom na to da klasični računari koji počivaju na tranzistorima dostižu fizičke granice svog daljnjeg razvitka, kvantni računari u ovom trenutku deluju kao jedan od najvažnijih budućih pravaca razvoja računarstva, uprkos velikim teškoćama na koje taj razvoj nailazi (među najvećima- izolovanost kvantnog računara od spoljnog sveta).}}
Na mnogim mestima paragrafi sadrže samo jednu rečenicu. Potrebno je restrukturiranje paragrafa u smislenije celine. \\
\odgovor{Izvršeno restruktuiranje pojedinih paragrafa koji su bili nepotrebno fragmentisani, tamo gde, po našem mišljenju, samostalna rečeni ne pripada, strukturalno, okružujućim paragrafima - uvodne rečenice (pod)poglavlja, rečenice koje predstavljaju zasebnu tematsku celinu - korekcije nisu vršene}

\section{Sitne primedbe}
3. strana:\\
-  Ponavljenje fraze Da bi smo i potrebno restrukturiranje paragrafa. 
\odgovor {Pomenuta greška je ispravljena.}\\
7. strana:\\
- U poglavlju 2.2.3. naslove faza bi trebalo istaći u odnosu na ostatk teksta (npr. bold ili slično)
\odgovor {Naslovi faza su istaknuti bold tekstom}

9. strana:\\
- greška: ugnežđene umesto ugneždene
\odgovor {Pomenuta greška je ispravljena.}


\section{Provera sadržajnosti i forme seminarskog rada}
% Oдговорите на следећа питања --- уз сваки одговор дати и образложење

\begin{enumerate}
\item Da li rad dobro odgovara na zadatu temu?\\
Rad dobro predstavlja osnove kvantnog računarstva i funkcionalne programske jezike na primeru jezika Quipper.
\item Da li je nešto važno propušteno?\\
Trebalo da se posveti više pažnje benefitu i prednostima kvantnog računarstva.
\odgovor {Potencijalni benefiti kvantnog računarstva su kompleksna tema u koju nismo hteli da dublje zalazimo iz prostornih razloga (već imamo primedbu na obim rada)}
\item Da li ima suštinskih grešaka i propusta?\\
Nisu primećeni značajniji propusti.
\item Da li je naslov rada dobro izabran?\\
Naslov nije formulisan, ostavljen je naslov teme.
\odgovor { Naslov rada je izmenjen. }
\item Da li sažetak sadrži prave podatke o radu?\\
Sažetak ne sadrži dovoljno informacija.
\odgovor {Mišljenja smo da sažetak treba da sadrži samu suštinu rada i da ne zalazi u detalje, što i jeste slučaj.}
\item Da li je rad lak-težak za čitanje?\\
Prvi deo rada koji se odnosi na kvantno računarstvo je lak za čitanje dok je drugi deo rada koji se odnosi na funkcionalni jezik Quipper teži za čitanje. 
\odgovor {Trudili smo se da rad bude što jednostavniji, ali da pri tom ne žrtvujemo tačnost podataka koji se navode.}
\item Da li je za razumevanje teksta potrebno predznanje i u kolikoj meri?\\
Potrebno je znanje osnovnih koncepata klasičnog računarstva i programskih jezika.
\item Da li je u radu navedena odgovarajuća literatura?\\
U radu je navedena odgovarajuća literatura ali za ovakvu temu je potrebno više izvora.
\odgovor {Namerno smo se ograničili na relativno malo literature jer je u pitanju tema o kojoj postoji dosta spekulacija i izvora koji nisu dovoljno pouzdani.}
\item Da li su u radu reference korektno navedene?\\
Navedene reference su korektne ali fale reference u prvom delu rada koji se odnosi na kvantno računarstvo. 
\odgovor{ Informacije u uvodnom delu rada su opšte prirode, navedena referenca ([3]) iscrpno pokriva oblast kojom se uvod bavi }
\item Da li je struktura rada adekvatna?\\
Osim što radu fali uvod, strukura je adekvatna.
\odgovor{Pretpostavljamo da recenzent misli na sažetak koji nismo hteli da nepotrebno proširujemo}
\item Da li rad sadrži sve elemente propisane uslovom seminarskog rada (slike, tabele, broj strana...)?\\
Rad ne sadrži tabelu i ne bi trebalo da prelazi 8 strana.
\odgovor {Naša je prvobitna ideja bila da ovo bude čisto teorijski rad, jer je vrlo teško napisati nešto smisleno na datu temu u obimu koji bi bio manji od našeg. Što se tiče tabele jednostavno nismo našli odgovarajuće podatke koje bi mogli prikazati u obliku tabele. }
\item Da li su slike i tabele funkcionalne i adekvatne?\\
Slika je adekvatna i dobrog kvaliteta ali bi trebalo da bude manje veličine, nepotrebno zauzima veći deo strane.
\odgovor {Veličina slike je smanjena.}
\end{enumerate}

\section{Ocenite sebe}
% Napišite koliko ste upućeni u oblast koju recenzirate: 
% a) ekspert u datoj oblasti
% b) veoma upućeni u oblast
% c) srednje upućeni
% d) malo upućeni 
% e) skoro neupućeni
% f) potpuno neupućeni
% Obrazložite svoju odluku

Veoma sam upućen u oblast.

\chapter{Drugi recenzent \odgovor{--- ocena: 2} }

\section{O čemu rad govori?}
% Напишете један кратак пасус у којим ћете својим речима препричати суштину рада (и тиме показати да сте рад пажљиво прочитали и разумели). Обим од 200 до 400 карактера.
U radu se govori o kvantnim računarima, sa akcentom na funkcionalne programske jezike za njih. 
Prvi deo je uvod, vrlo lepo i jednostavno obrađen, a kasnije se govori o 
funkcionalnim jezicima uopšte kao i za kvantne računare, poput QFC, QPL i Quipper-a. 

\section{Krupne primedbe i sugestije}
% Напишете своја запажања и конструктивне идеје шта у раду недостаје и шта би требало да се промени-измени-дода-одузме да би рад био квалитетнији.
Tema je veoma zanimljiva, posebno jer se vrši osvrt na funkcionalne jezike, koji su danas manje 
primenjeni u praksi, i vrlo je fino i korektno obrađena. Nema velikih zamerki.

\section{Sitne primedbe}
% Напишете своја запажања на тему штампарских-стилских-језичких грешки
Postoje određene sitne greške u kucanju koje bi trebalo ispraviti (poput kvadarata u odeljku 1.2.2 
u prvoj formuli). Možda bi samo, radi potpunosti rada, bilo lepo dodati poneki primer, u vidu koda.\\
\odgovor {Ispravljena greška nedostajućeg kvadrata u pomenutom odeljku; u samom radu nalazi se jedan jednostavan primer, a ostatak nije bio predviđen da se navodi u teorijskom delu. Propratni primeri za jezik Quipper su uključeni u konačnu verziju rada}
\section{Provera sadržajnosti i forme seminarskog rada}
% Oдговорите на следећа питања --- уз сваки одговор дати и образложење

\begin{enumerate}
\item Da li rad dobro odgovara na zadatu temu?\\
Da. Uvod je lepo urađen, opisani su fukcionalni jezici, sve je na mestu.

\item Da li je nešto važno propušteno?\\
Verovatno ne. Mislim da je suština obuhvaćena i sve je objašnjeno lepo i jednostavno.

\item Da li ima suštinskih grešaka i propusta?\\
Ne, suštinskih. Sitnih propusta uvek ima, ali suština je tu. 

\item Da li je naslov rada dobro izabran?\\
Da. Naslov je intuitivan, i iz njega se jasno vidi o čemu se govori.

\item Da li sažetak sadrži prave podatke o radu?\\
Da. Sadrži prave podatke, vrlo jasan i kratak. 

\item Da li je rad lak-težak za čitanje?\\
Relativno lako. Vidi se da su kolege uspele da približe tematiku čitaocu.

\item Da li je za razumevanje teksta potrebno predznanje i u kolikoj meri?\\
I da i ne. Predznanje koristi, ali za osnovno upoznavanje, nije potrebno.

\item Da li je u radu navedena odgovarajuća literatura?\\
Da. Odeljak sa literaturom je izdvojen i lepo odrađen.

\item Da li su u radu reference korektno navedene?\\
Da. Reference su povezane sa podacima u tekstu gde se koriste.

\item Da li je struktura rada adekvatna?\\
Da. Jasno je i lako za snalaženje.

\item Da li rad sadrži sve elemente propisane uslovom seminarskog rada (slike, tabele, broj strana...)?\\
Da, sadrži. Potrebno je samo pogledati rad. 

\item Da li su slike i tabele funkcionalne i adekvatne?\\
Da. Krajnje. Za sve postoji kratko objašnjenje, fino definisano.

\end{enumerate}

\section{Ocenite sebe}
% Napišite koliko ste upućeni u oblast koju recenzirate: 
% a) ekspert u datoj oblasti
% b) veoma upućeni u oblast
% c) srednje upućeni
% d) malo upućeni 
% e) skoro neupućeni
% f) potpuno neupućeni
% Obrazložite svoju odluku

Malo sam upućen u ovu tematiku. Prosto, o kvantnim računarima sam čuo tu i tamo, 
kratko se zaustavim da čujem osnove i onda se bacim na ono što se danas koristi. 
Kvantni računari možda jesu budućnost ali to je tehnologija koja se tek razvija. 



\chapter{Treći recenzent \odgovor{--- ocena:} }

\section{O čemu rad govori?}
Rad govori o kvantnom programiranju, dajući koristan uvod u kvantno računarstvo i funkcionalno programiranje, opisujući efekte kvantne mehanike i njene prednosti pri konstrukciji računara, pojmove i koncepte kvantnog računarstva i osnove funkcionalne paradigme. Rad se zatim fokusira na funkcionalno kvantno programiranje kroz dva opširna primera programskih jezika.
% Напишете један кратак пасус у којим ћете својим речима препричати суштину рада (и тиме показати да сте рад пажљиво прочитали и разумели). Обим од 200 до 400 карактера.

\section{Krupne primedbe i sugestije}
Jedina velika zamerka jeste nenavođenje referenci u celom radu, nego samo na nekim mestima. \\
\odgovor { Uvažili smo ovu sugestiju i dodali reference na nekoliko najvažnijih mesta, jer da smo bili potpuno dosledni reference bismo prečesto navodili. }

% Напишете своја запажања и конструктивне идеје шта у раду недостаје и шта би требало да се промени-измени-дода-одузме да би рад био квалитетнији.

\section{Sitne primedbe}
U radu postoji nekoliko malih gramatičkih  i štamparskih grešaka, a na više mesta i nepotrebne podele na pasuse. Takođe, bilo bi dobro prilagoditi terminologiju čitaocima i podneblju, kao što je slučaj sa terminima "nus-pojava" (bočni efekat) i "kjubit" (kubit).\\
\odgovor {Podela na pasuse je izmenjena; koliko znamo, termin nuspojava je reč srpskog jezika, tako da ne bi trebalo biti problema sa razumevanjem te reči. Ali smo kasnije saznali da je njen pravilan zapis bez crtice, tako da smo to izmenili. Što se tiče termina 'kjubit', postoji mogućnost da se ona navodi onako kako se piše na engleskom jeziku (za šta, po nama, nema nikakve potrebe, pošto je rad pisan na srpskom) ili da se piše onako kako se čita (što je rešenje koje smo mi odabrali).}

% Напишете своја запажања на тему штампарских-стилских-језичких грешки


\section{Provera sadržajnosti i forme seminarskog rada}
% Oдговорите на следећа питања --- уз сваки одговор дати и образложење

\begin{enumerate}
\item Da li rad dobro odgovara na zadatu temu?\\
Da, rad dobro, u odgovarajućem obimu i u skladu sa temom opisuje funkcionalno kvantno programiranje.
\item Da li je nešto važno propušteno?\\
Nije propušteno ništa važno.
\item Da li ima suštinskih grešaka i propusta?\\
U radu nema suštinskih grešaka.
\item Da li je naslov rada dobro izabran?\\
Naslov rada nije dobro izabran jer je to naslov teme, a i pored toga nije prikladan.\\
\odgovor {Izmenili smo naslov rada.} 
\item Da li sažetak sadrži prave podatke o radu?\\
Da, sažetak sadrži prave podatke o radu, mada bi mogao biti obimniji.\\
\odgovor {Mišljenja smo da nema potrebe da sažetak bude detaljniji. }
\item Da li je rad lak-težak za čitanje?\\
Rad nije težak za čitanje.
\item Da li je za razumevanje teksta potrebno predznanje i u kolikoj meri?\\
Za rad je potrebno osnovno informatičko i solidno matematičko znanje.\\
\odgovor {Kao što smo već pomenuli, trudili smo se da rad napišemo što jednostavnije, ali nismo hteli da zbog toga žrtvujemo tačnost podataka. Takođe, nismo sigurni na šta konkretno recenzent referira kad kaže da je potrebno solidno matematičko predznanje.}
\item Da li je u radu navedena odgovarajuća literatura?\\
U radu jeste navedena odgovarajuća literatura, ali ne u neophodnom obimu.\\
\odgovor {Već smo dali odgovar na ovu kritiku u ranijoj recenziji, ali smo je delom i uvažili, te smo literaturu u nekoj meri proširili.}
\item Da li su u radu reference korektno navedene?\\
Neophodno je bolje navođenje referenci u radu, naročito zbog toga što nekim segmentima nedostaju reference.\\
\odgovor {Reference su dodate na još nekoliko značajnih mesta u radu.}

\item Da li je struktura rada adekvatna?\\
Možda bi bilo bolje da se deo obuhvaćen Uvodom podeli u zasebne celine, inače je struktura adekvatna.
\odgovor{Nismo sigurni kako bi dalje podelili delove uvoda, grupisani pojmovi su povezani i nismo sigurni da bi to doprinelo čitljivosti rada}
\item Da li rad sadrži sve elemente propisane uslovom seminarskog rada (slike, tabele, broj strana...)?\\
Rad ne sadrži tabele, ali sadrži slike, a broj strana je veći za jednu.\\
\odgovor {Naša je prvobitna ideja bila da ovo bude čisto teorijski rad, jer je vrlo teško napisati nešto smisleno na datu temu u obimu koji bi bio manji od našeg. Što se tiče tabele jednostavno nismo našli odgovarajuće mesto i podatke koje bi mogli prikazati u obliku tabele.}
\item Da li su slike i tabele funkcionalne i adekvatne?\\
Slika jeste adekvatna, a rad ne sadrži tabele.\\
\odgovor {Obrazloženje dato u prethodnom odgovoru.}
\end{enumerate}

\section{Ocenite sebe}
% Napišite koliko ste upućeni u oblast koju recenzirate: 
% a) ekspert u datoj oblasti
 %b) veoma upućeni u oblast
 c) srednje upućeni
% d) malo upućeni 
% e) skoro neupućeni
% f) potpuno neupućeni
% Obrazložite svoju odluku

U oblast kvantnog programiranja sam dobro upućen, imajući u vidu da sam je istraživao nedavno, ali u oblast funkcionalnog kvantnog programiranja sam slabije upućen.


\chapter{Četvrti recenzent \odgovor{--- ocena: 5 } }%primedbe su krajnje konkretne u i većini slučajeva odgovarajuće 

\section{O čemu rad govori?}
% Напишете један кратак пасус у којим ћете својим речима препричати суштину рада (и тиме показати да сте рад пажљиво прочитали и разумели). Обим од 200 до 400 карактера.
U početnim poglavljima autori objašnjavaju šta je kvantno računarstvo kao i koji su izazovi u kvantnom računarstvu. Kasnije, uz kratak pregled postojećih funkcionalnih jezika za kvantne računare, navode osobine o izabranim programskim jezicima.

\section{Krupne primedbe i sugestije}
% Напишете своја запажања и конструктивне идеје шта у раду недостаје и шта би требало да се промени-измени-дода-одузме да би рад био квалитетнији.
Zaključak priča o budućim revizijama. Taj deo bi trebalo da se izbaci, nema potrebe za pisanjem šta treba da se uradi. \\
\odgovor {Slažemo se sa sugestijom, tako da smo taj deo izbacili iz zaključka.}

\section{Sitne primedbe}
% Напишете своја запажања на тему штампарских-стилских-језичких грешки
\begin{enumerate}
\item Na 4. strani ceo prvi pasus treba da se sredi.
Rečenica ``Neki od opisanih funkcionalnih jezika (Quipper, ?)...`` treba da se zamini odgovarajućom.
Uvođenje novog pojma QRAM nema zatvorenu zagradu.\\
\odgovor {Slažemo se, uneli smo pomenute izmene.}

\item Na više mesta postoje greške u slovima, npr. umesto slova ``đ`` piše ``dj``.
\odgovor {Kritika je tačna, pomenute greške smo ispravili.}
\item U poglavlju 1.3 ostavljeno je mesto za referencu ali referanca nije ubačena.
\odgovor {Kritika je tačna, referenca je ubačena.}
\item Poglavlje 2 nema tekst, odmah se kreće sa poglavljem 2.1.

\item U poglavlju 2.1 postoji referenca za koju ne postoji tekst??
%ovde je dodavanje reference valjda dovoljno da reši problem. Mislio sam da je već bila tamo ... 
\item Na strani 6 postoji u tekstu referenca na sliku no ona nije uvezana.\\
\odgovor {Pošto je navedena jedna slika u celom radu, nismo videli potrebu da preciziramo koja je slika u pitanju, ali ipak smo uvažili i ovu sugestiju.}
\item Na strani 6 u blizini formule rečenica kreće sa ``Npr., ...``. Mislim da je lepše napisati to ``Na primer, ...``. U istoj rečenici tačku treba staviti pored formule ili je ne stavljati uopšte. Mislim da ne treba da krene novi red sa tačkom.
\odgovor {Prva sugestija je stvar ukusa, ali smo je ipak uvažili. A što se tiče tačke, potpuno se slažemo, to je detalj koji nam je promakao.}
\item Na strani 6, zbog korišćenja znakova \" za navodnike, umesto ``discard`` dobijena je reč điscard. Koristiti iskošene jednostruke navodnike (pored tastera za 1) za navodnike.
\odgovor {Još jedan detalj koji nam je promakao, potpuno se slažemo, promena je izvršena.}
\end{enumerate}

\section{Provera sadržajnosti i forme seminarskog rada}
% Oдговорите на следећа питања --- уз сваки одговор дати и образложење

\begin{enumerate}
\item Da li rad dobro odgovara na zadatu temu?\\
Da, autori nisu skrenuli sa teme, obuhvatili su najbitnije detalje.

\item Da li je nešto važno propušteno?\\
Nije ništa važno propušteno

\item Da li ima suštinskih grešaka i propusta?\\
Nema. 

\item Da li je naslov rada dobro izabran?\\
Naslov rada je naziv teme. Ako autori nisu mogli da smisle kreativniji naslov mogli su barem broj ``2`` da izbace. \\
\odgovor {Nismo ni znali da naslov teme ne treba biti naslov rada. U svakom slučaju, promenili smo naslov rada.}

\item Da li sažetak sadrži prave podatke o radu?\\
Da. Sažetak sadrži tačno opisuje ono o čemu autori govore u radu.

\item Da li je rad lak-težak za čitanje?\\
Rad je lak za čitanje jer su osnovni pojmovi lepo objašnjeni.

\item Da li je za razumevanje teksta potrebno predznanje i u kolikoj meri?\\
Za čitanje rada nije potrebno predznanje o kvantnom računarstvu. 

\item Da li je u radu navedena odgovarajuća literatura?\\
Navedena je odgovarajuća literatura.

\item Da li su u radu reference korektno navedene?\\
Reference su konkretno navedene.

\item Da li je struktura rada adekvatna?\\
Struktura rada je adekvatna.

\item Da li rad sadrži sve elemente propisane uslovom seminarskog rada (slike, tabele, broj strana...)?\\
Rad ne sadrži nikakve tabele.
\odgovor { Jednostavno nismo našli odgovarajuće podatke koje bi mogli prikazati u obliku tabele.}

\item Da li su slike i tabele funkcionalne i adekvatne?\\
Slika je na pravmom mestu, lepo opisuje ono što je napisano u tekstu
\end{enumerate}

\section{Ocenite sebe}
% Napišite koliko ste upućeni u oblast koju recenzirate: 
% a) ekspert u datoj oblasti
% b) veoma upućeni u oblast
% c) srednje upućeni
% d) malo upućeni 
% e) skoro neupućeni
% f) potpuno neupućeni
% Obrazložite svoju odluku
Potpuno neupućen. Za kvantno računarstvo sam čuo u par navratna ranije, no nikada nisam ništa više istraživao o toj grani računarstva.


\chapter{Dodatne izmene}
%Ovde navedite ukoliko ima izmena koje ste uradili a koje vam recenzenti nisu tražili. 

\end{document}


