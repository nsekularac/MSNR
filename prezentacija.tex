%%%%%%%%%%%%%%%%%%%%%%%%%%%%%%%%%%%%%%%%%
% Beamer Presentation
% LaTeX Template
% Version 1.0 (10/11/12)
%
% This template has been downloaded from:
% http://www.LaTeXTemplates.com
%
% License:
% CC BY-NC-SA 3.0 (http://creativecommons.org/licenses/by-nc-sa/3.0/)
%
%%%%%%%%%%%%%%%%%%%%%%%%%%%%%%%%%%%%%%%%%

%----------------------------------------------------------------------------------------
%	PACKAGES AND THEMES
%----------------------------------------------------------------------------------------

\documentclass[12pt,hyperref={unicode}]{beamer}

\usepackage[utf8x,utf8]{inputenc} % make weird characters work
\usepackage[serbian]{babel}
\usepackage{physics}

\mode<presentation> {

% The Beamer class comes with a number of default slide themes
% which change the colors and layouts of slides. Below this is a list
% of all the themes, uncomment each in turn to see what they look like.

\usetheme{Madrid}

% As well as themes, the Beamer class has a number of color themes
% for any slide theme. Uncomment each of these in turn to see how it
% changes the colors of your current slide theme.

\setbeamertemplate{footline} % To remove the footer line in all slides uncomment this line
%\setbeamertemplate{footline}[page number] % To replace the footer line in all slides with a simple slide count uncomment this line

%\setbeamertemplate{navigation symbols}{} % To remove the navigation symbols from the bottom of all slides uncomment this line
}

\usepackage{graphicx} % Allows including images
\usepackage{booktabs} % Allows the use of \toprule, \midrule and \bottomrule in tables

%----------------------------------------------------------------------------------------
%	TITLE PAGE
%----------------------------------------------------------------------------------------

\title[Short title]{Uvod u kvantno računarstvo sa osvrtom na funkcionalne programske jezike za kvantno programiranje} % The short title appears at the bottom of every slide, the full title is only on the title page

\author{Aleksandar Ćurković, Nemanja Šekularac} % Your name
\institute % Your institution as it will appear on the bottom of every slide, may be shorthand to save space
{
Matematički fakultet \\ % Your institution for the title page
\medskip
\textit{nsekularac@gmail.com, curkovical@gmail.com} % Your email address
}
\date{11. maj 2016} % Date, can be changed to a custom date

\begin{document}

\begin{frame}
\titlepage % Print the title page as the first slide
\end{frame}

\begin{frame}
\frametitle{Pregled} % Table of contents slide, comment this block out to remove it
\tableofcontents % Throughout your presentation, if you choose to use \section{} and \subsection{} commands, these will automatically be printed on this slide as an overview of your presentation
\end{frame}

%----------------------------------------------------------------------------------------
%	PRESENTATION SLIDES
%----------------------------------------------------------------------------------------

%------------------------------------------------
\section{Uvod} % Sections can be created in order to organize your presentation into discrete blocks, all sections and subsections are automatically printed in the table of contents as an overview of the talk
%------------------------------------------------

\subsection{Kvantno računarstvo}% A subsection can be created just before a set of slides with a common theme to further break down your presentation into chunks


\begin{frame}
\frametitle{Kvantno računarstvo}
\begin{itemize}

\item{Efekti kvantne mehanike}
	\begin{itemize}
    \item{superpozicija}\\
	čestica postoji u više stanja istovremeno, pre merenja
    \item{spletenost}\\
	rezultat merenja jedne čestice determiniše rezultat merenja druge čestice, bez obzira na njihovu udaljenost
    \end{itemize}
\item{Kvantni računar je računar koji u svom radu direktno koristi efekte kvantne mehanike za operacije nad podacima}
\end{itemize}

\end{frame}

%------------------------------------------------

\begin{frame}
\frametitle{Kvantno računarstvo}
\begin{itemize}
\item{Kubit}
\begin{itemize}
    \item{najmanja jedinica izračunavanja u kvantnom računarstvu}\\
    \item{osnovna stanja, $\ket{1}$ i $\ket{0}$.}\\
    \end{itemize}
\item{Kvantni registar - skup kubita}
\item{Kvantne kapije}
\begin{itemize}
    \item{osnovne operacije nad kubitima}
    \item{reverzibilne operacije predstavljene unitarnim operatorima}
    \end{itemize}
\item{Kvantno kolo - sekvenca kvantnih kapija}
\item{}
\end{itemize}
\end{frame}

\subsection{Funkcionalno programiranje}
%---------------------------------------------

\begin{frame}
\frametitle{Funkcionalno programiranje}
\begin{itemize}
\item{Programska paradigma u kojoj je osnovni način izračunavanja primena funkcija na argumente}\cite{p4}
\item{Funkcije višeg reda}
    \begin{itemize}
    \item{za argumente uzimaju druge funkcije ili kao svoj rezultat vraćaju funkciju }
    \end{itemize}
\item{Odsustvo nuspojava}
  	\begin{itemize}
	\item{više poziva iste funkcije uvek vraća isti rezultat za iste argumente poziva}
  	\end{itemize}
\end{itemize}

\end{frame}

%------------------------------------------------
\section{QFC}
%------------------------------------------------

\begin{frame}
\frametitle{QFC (Quantum Flow Charts)}
\begin{itemize}
\item{QPL(Quantum Programming Language)}
	\begin{itemize}
	\item{ekvivalentan QFC-u}
    \item{umesto grafičke tekstualna reprezentacija}
	\end{itemize}
\item{''kvantni podaci, klasična kontrola''}\cite{p1}
\item{program se predstavlja u obliku grafikona protoka}
\item{podržava petlje, rekurziju i procedure, kao i strukture podataka kao što su liste i drveta}
\item{svaka grana obeležena dostupnim kubitovima, čvorovi označavaju operacije}
\item{dodatna pravila: new, discard, initial i permute}
\end{itemize}
\end{frame}

%------------------------------------------------
\section{Quipper}
%------------------------------------------------

\begin{frame}
\frametitle{Quipper}
\begin{itemize}
\item{Skalabilni, funkcionalni jezik višeg nivoa za kvantno programiranje \cite{p3}}
\item{Kao model kvantnog računara koristi se kvantna memorija sa nasumičnim pristupom (eng.~\emph{Quantum Random Access Memory, Quantum RAM, QRAM}})
\item{Implementiran kao ugnežden jezik u Haskellu}
\item{Ugrađen simulator kvantnog računara}
\item{Uključena biblioteka aktuelnih kvantnih algoritama}
\end{itemize}
\end{frame}
%------------------------------------------------
\begin{frame}
\frametitle{Quipper, osobine jezika}
\begin{itemize}
\item{Jezik za opis kvantnih kola sa podrškom za pomešano proceduralno i deklarativno programiranje}
\item{Kubiti kao promenljive, kapije kao funkcije, operacije višeg nivoa nad celim funkcijama}
\item{Mešana klasično/kvantna kola, kontrolisano uvođenje i dealokacija dodatnih i kontrolnih promenljivih}
\item{Tri faze izvršenja: kompilacija, generisanje kola, izvršenje kola}
\end{itemize}
\end{frame}

%------------------------------------------------
\section{Reference}
%------------------------------------------------

\begin{frame}
\frametitle{Reference}
\footnotesize{
\begin{thebibliography}{99} % Beamer does not support BibTeX so references must be inserted manually as below
\bibitem[Selinger, 2004]{p1} Peter Selinger (2004)
\newblock Towards a Quantum Programming Language
\setbeamertemplate{bibliography item}[book]
\bibitem{p2} Nielsen, Michael A. and Chuang, Isaac L. (2010)
\setbeamertemplate{bibliography item}[article]
\newblock Quantum Computation and Quantum Information: 10th Anniversary Edition
\bibitem{p3} Alexander S. Green and Peter LeFanu Lumsdaine and Neil J. Ross and Peter Selinger and Benoît Valiron (2013)
\newblock Quipper: A Scalable Quantum Programming Language
\bibitem[Michaelson, 2011]{p4} Greg Michaelson (2011)
\newblock An introduction to functional programming through lambda calculus
\end{thebibliography}
}
\end{frame}

%----------------------------------------------------------------------------------------

\end{document} 
